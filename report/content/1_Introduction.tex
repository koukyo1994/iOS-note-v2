\section{Introduction}

Handwriting has played a very important role as a means of human knowledge preservation
or communication from the invention of letters to the modern times.
Its role has been reduced with the spread of printing and computers,
but in many parts of our lives, we still have the opportunity to write and read handwriting.
For example, Students in a classroom still prefer to use paper and pen or electronic tablet
to store knowledge, mathematical formulas, graphs, etc.,
rather than taking notes using keyboard and computer.
On the other hand, in computer work such as searching or document writing,
input is almost always performed with keyboard, and input by handwriting is rare.
This is, of course, because computer cannot handle handwritten input as text as it is,
but also because keyboard is more efficient for inputting characters.
With a keyboard, we can enter characters with less movement compared to handwriting.
Besides, there are also assisting functions for keyboard input such as copy and paste,
searching, or formatting. Since keyboard input has these overwhelming advantages,
people usually try to use keyboard even for tasks which are actually easier to be done by hand,
or to use keyboard for writing and use tablet device for those tasks that are not achievable with keyboard.
However, with the spread of tablet devices in recent years, this situation is showing signs of change.
On these devices, input work that keyboards are not good at, such as drawing illustrations and
writing mathematical formulas, can be easier. In addition, applications
\footnote{https://www.nebo.app/}\footnote{http://mazec.jp/} have appeared that can handle handwriting
as text format by applying Handwriting Recognition technology that has been studied for many years.
Thanks to these technological improvements, handwriting is becoming an available option for inputting text,
but it is not yet as efficient as a keyboard is. One of the reason for this is the lack of input assistance.
Therefore, in this work, we propose an application that combines Handwriting Recognition and auto-completion
to assist handwriting and to enable it to be treated as text format on devices.

Our application consists of two elements: Handwriting Recognition and auto-completion.
Handwriting Recognition (HWR) is a technology that
interprets handwritten inputs from sources such as papers,
photographs, electronic tablets, and other devices into a format
that can be easily handled by computers.
In most of the modern work of HWR, the process pipeline is divided into Handwritten Text Detection
and Handwritten Text Recognition~\cite{long2018scene}. The processing of text detection
tends to be complicated\cite{long2018scene}, and it takes a considerable amount of time to
perform processing on the entire input image. It is known
\footnote{http://glinden.blogspot.com/2006/11/marissa-mayer-at-web-20.html} that longer processing time
in applications that require a quick response, such as auto-completion,
can seriously degrade the user experience. Therefore we propose a method to avoid performing
Text Detection using a simple heuristic: Region of Interest (ROI) detection.

Auto-completion is a program that predicts the rest of the word a user is typing based on the letters entered
or the recent word pairs~\cite{darragh1990reactive}. It is widely used in web browsers, in email programs,
in source code editors or in word processors. If the correct word is included in the predicted words list,
the user can save time to enter the rest of the word. In this work, to make the process simple,
we used forward matching algorithm for this purpose.

In the following section, typical approaches of HWR and related works are introduced.
Section \ref{section:methods} provides technical details of the approach to the problem addressed in
this report. Section \ref{section:result-discussion} describes the result of the approach and discuss on that.
Section \ref{section:conclusion} concludes this report with future prospects.
