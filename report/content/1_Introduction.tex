\section{Introduction}

Handwriting recognition (HWR) is a field which study and develop algorithms to
interpret handwritten inputs into a format that can be easily handled by computers
from sources such as papers, photographs, electronic tablets, and other devices.
HWR can be roughly devided into two appproaches:
online approach and offline approach~\cite{plamondon2000online}.\\
While online approach uses information on the trajectory of the pen tip
obtained from a special pen for classification, offline method uses
optically scanned images as input and performs recognition using computer vision techniques.
This report focuses only on offline approach, and tackle on the recognition problem
using Convolutional Neural Network (CNN), which have shown a remarkable development in recent years.\\

% TODO: add some references of these applications
HWR is a field that has been studied for a long time, and many applications have already been made.
However, in the case of tablet devices that have spread rapidly in recent years, not so many applications
have been created even after APIs to incorporate with pattern recognition algorithms on device are published
by developpers of those devices. Although some applications\footnote{https://www.nebo.app/} have achieved very good
result on normal handwritten text recognition, it is not the case when elements in other domains
such as handwritten illustlations are mixed in addition to sentences.
Also, APIs published by the manufacturer of those devices are usually black-boxes and poorly extensible.
The main target of these APIs is printed text, and when applied to handwritten characters,
performance may not be as high as expected.

I therefore endeavor to recognize handwritten documents on device, especially those which contain not only text,
but also handwritten illustlations or mathematical formulas. Since this type of documents are very common
in our daily life, the success of the project can potentially bring the fusion of digital
technology and the long-standing human skills of handwriting. \\

In the following section, typical approaches of HWR and related works are introduced.
Section three provides technical details of the approach to the problem addressed in
this report. Section four describes the result of the approach and discuss on that.
Section five concludes this report with future prospects.\\
