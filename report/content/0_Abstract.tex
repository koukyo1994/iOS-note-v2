Handwriting is still in strong demand as a way of communication and saving knowledge
even in the era of new technology. Although writing sentences is easier with a keyboard,
handwriting is better at expressing illustrations and formulas. It is also attractive
that the degree of freedom of writing is larger than keyboard input.
Despite these advantages, keyboard input is widely used in computer-based processing because
handwriting input is so diverse that it is not in a format suitable for computer processing.
The major drawbacks of handwriting input resulting from this characteristic are its low
portability and little assistance in inputting. To address these drawbacks,
Handwriting Recognition, technology which converts handwritten input into a form that
can be handled by a computer, has been studied for a long time.
Driven by the rise of electronic pens and tablets, we have more opportunities than before to
write by hand with these devices.
Therefore, the need for Handwriting Recognition on electronic devices is also increasing recently.
However, Handwriting Recognition applications on devices are still evolving
and the true value of electronic tablets as a medium of
handwriting has not yet been exploited enough.
This report focuses on application of Handwriting Recognition on iOS device and proposes
auto-completion using the recognition result as an assistance for handwriting input
to alleviate the inconvenience of handwriting. We show that our auto-completion can
reduce the number of characters that need to be actually entered to write the word the user want
to write, and thus reduce the lack of input assistance. We also propose a technique to significantly reduce
the amount of computation required, that is also an important aspect of processing on device.
We provide two metrics to measure the
performance of auto-completion, and compare the performance difference between the case
using the text recognition API provided by iOS and the case using the Handwriting Recognition model
designed to reduce the amount of computation.
