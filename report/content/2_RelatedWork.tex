\section{Related Work}

The problem settings of this project can be positioned as one variant of Scene Text Detection/Recognition, which
is a field to study algorithms to extract and recognize text information written in natural images.
Due to the recent development of Neural Networks technology,
much research has been done in this field to this day~\cite{long2018scene}.\\
Except for few methods\cite{liu2018fots}\cite{lyu2018mask}, most approaches of Scene Text Detection/Recognition
separate detection and recognition and perform stepwise inference.

\subsection{Detection}

Scene Text Detection can be subsumed under general object detection, therefore those methods usually follow
the same procedure of object detection, which is dichotomized as one-stage methods and two-stage ones~\cite{liu2018deep}.\\

Object detection methods after the emergence of Object detection methods are

\subsection{Recognition}

Some text recognition algorithms devide the task into character segmentation and character recognition~\cite{bissacco2013photoocr}\cite{phan2011gradient}.
Character segmentation is considered as the most challenging part of scene text recognition, and may affect
overall accuracy. It is especially difficult to segment connected characters such as cursive.
Therefore some techniques which do not rely on character segmentation have been developped so far.
This report introduces a method called Connectionist Temporal Classification (CTC)~\cite{graves2006connectionist}.

CTC was first introduced to handle sequence labeling of arbitrary length,
requiring no pre-segmented training data.
