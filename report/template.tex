%%%%%%%%%%%%%%%%%%
%  placeholders  %
%%%%%%%%%%%%%%%%%%

\newcommand{\uni}{Technical University of Munich}
\newcommand{\department}{TUM Department of Informatics}
\newcommand{\chair}{Chair for Applied Software Engineering}
\newcommand{\city}{Munich}
\newcommand{\country}{Germany}

% TODO: Replace with your information
\title{Title}
\newcommand{\authorname}{YOUR NAME}
\newcommand{\email}{email address or ORCID}

\input{settings}
    
\begin{document}

\author{
	\IEEEauthorblockN{\authorname}
	\IEEEauthorblockA{\textit{\department, \chair} \\
	\textit{\uni}\\
	\city, \country \\
	\email}}

\maketitle

\begin{abstract}
Handwriting is still in strong demand as a way of communication and saving knowledge
even in the era of new technology.

\end{abstract}

%%%%%%%%%%%%%%%%%%%%%%%%%
%  General Information  %
%%%%%%%%%%%%%%%%%%%%%%%%%

% Keep the slides from the Kick-Off Meeting in mind.
% The goal of the seminar is to take a look at applications of machine learning in real world scenarios or with real world applications.
% We want to take theoretical concepts and new approaches and apply them to existing or emerging problems.
% It is important that you document your process, collect citations, the process how you got to the results is as valuable as the end result.
% The seminar be about 6 pages (+ bibliography) long. It should summarize your literature research and related work.
% - Related practical work e.g. frameworks, tools, applications
% - Related academic literature such as papers and books
% - Showcases your proposed solution and prototype
% - Describe the implementation, challenges, and interesting technical details
% - Showcase why your prototype is a relevant for consumers or research
% - Details possible future work

\section{Introduction}

Handwriting has played a very important role as a means of human knowledge preservation
or communication from the invention of letters to the modern times.
Its role has been reduced with the spread of printing and computers,
but in many parts of our lives, we still have the opportunity to write and read handwriting.
For example, Students in a classroom still prefer to use paper and pen or electronic tablet
to store knowledge, mathematical formulas, graphs, etc.,
rather than taking notes using keyboard and computer.
On the other hand, in computer work such as searching or document writing,
input is almost always performed with keyboard, and input by handwriting is rare.
This is, of course, because computer cannot handle handwritten input as text as it is,
but also because keyboard is more efficient for inputting characters.
With a keyboard, we can enter characters with less movement compared to handwriting.
Besides, there are also assisting functions for keyboard input such as copy and paste,
searching, or formatting. Since keyboard input has these overwhelming advantages,
people usually try to use keyboard even for tasks which are actually easier to be done by hand,
or to use keyboard for writing and use tablet device for those tasks that are not achievable with keyboard.
However, with the spread of tablet devices in recent years, this situation is showing signs of change.
On these devices, input work that keyboards are not good at, such as drawing illustrations and
writing mathematical formulas, can be easier. In addition, applications
\footnote{https://www.nebo.app/}\footnote{http://mazec.jp/} have appeared that can handle handwriting
as text format by applying Handwriting Recognition technology that has been studied for many years.
Thanks to these technological improvements, handwriting is becoming an available option for inputting text,
but it is not yet as efficient as a keyboard is. One of the reason for this is the lack of input assistance.
Therefore, in this work, we propose an application that combines Handwriting Recognition and auto-completion
to assist handwriting and to enable it to be treated as text format on devices.

Our application consists of two elements: Handwriting Recognition and auto-completion.
Handwriting Recognition (HWR) is a technology that
interprets handwritten inputs from sources such as papers,
photographs, electronic tablets, and other devices into a format
that can be easily handled by computers.
In most of the modern work of HWR, the process pipeline is divided into Handwritten Text Detection
and Handwritten Text Recognition~\cite{long2018scene}. The processing of text detection
tends to be complicated\cite{long2018scene}, and it takes a considerable amount of time to
perform processing on the entire input image. It is known
\footnote{http://glinden.blogspot.com/2006/11/marissa-mayer-at-web-20.html} that longer processing time
in applications that require a quick response, such as auto-completion,
can seriously degrade the user experience. Therefore we propose a method to avoid performing
Text Detection using a simple heuristic: Region of Interest (ROI) detection.

Auto-completion is a program that predicts the rest of the word a user is typing based on the letters entered
or the recent word pairs~\cite{darragh1990reactive}. It is widely used in web browsers, in email programs,
in source code editors or in word processors. If the correct word is included in the predicted words list,
the user can save time to enter the rest of the word. In this work, to make the process simple,
we used forward matching algorithm for this purpose.

In the following section, typical approaches of HWR and related works are introduced.
Section \ref{section:methods} provides technical details of the approach to the problem addressed in
this report. Section \ref{section:result-discussion} describes the result of the approach and discuss on that.
Section \ref{section:conclusion} concludes this report with future prospects.

\section{Related Work}

The problem settings of this project can be positioned as one variant of Scene Text Detection/Recognition, which
is a field to study algorithms to extract and recognize text information written in natural images.
Due to the recent development of Neural Networks technology,
much research has been done in this field to this day~\cite{long2018scene}.\\
Except for few methods\cite{liu2018fots}\cite{lyu2018mask}, most approaches of Scene Text Detection/Recognition
separate detection and recognition and perform stepwise inference.

\subsection{Detection}

Scene Text Detection can be subsumed under general object detection, therefore those methods usually follow
the same procedure of object detection, which is dichotomized as one-stage methods and two-stage ones~\cite{liu2018deep}.\\

Object detection methods after the emergence of Object detection methods are

\subsection{Recognition}

Some text recognition algorithms devide the task into character segmentation and character recognition~\cite{bissacco2013photoocr}\cite{phan2011gradient}.
character segmentation is not currently widely used
since it can be a cause of errors when separating connected characters such as cursive,
but there is also a merit that it is only necessary to use a simpler character recognition method
instead of text recognition in the subsequent processing.

\section{Contributions}

\begin{itemize}
	\item Lead up to your contributions.
	\item Describe your research process.
	\item Start with theoretical work and work yourself to its applications in your reproach report.
	\item Document your implementation and solution to the problem described in previous sections.
	\item Discuss your data collection, training or implementation approach and highlight interesting technical details.
	\item Feel free to add more sections or subsections and rename existing sections e.g. the Contributions section as you need.
\end{itemize}
\section{Results \& Discussion}

\subsection{Region of Interest Detection}

It is desirable for region of interest detection to cut out the region which if focused by
the user. For example, when a user is writing a sentence,region of interest should be a word
written, and when a user is drawing an illustration, region of interest should be the whole
illustration drawn.

Since it is difficult to quantitatively measure the goodness of the region of interest detection,
only qualitative evaluation was performed this time.

When cutting out only the words that the user is writing in the text,
the region of interest detection worked almost without failure (Fig \ref{fig:cutting-word-region}).

\begin{figure}
    \centering
    \includegraphics[scale=0.6]{images/word_region.png}
    \caption{region of interest detection works well when cutting out a word which is being written}
    \label{fig:cutting-word-region}
\end{figure}

On the other hand, when cutting out a handwritten illustration,
only a part of the illustration may be cut out if the lines constituting
the illustration are not spatially close to each other (Fig \ref{fig:failure}).

\begin{figure}
    \centering
    \includegraphics[scale=0.4]{images/illustration.png}
    \includegraphics[scale=0.9]{images/failure.png}
    \caption{Left: example of the case ROI detection worked well on illustration. Right: example of failure case of ROI detection}
    \label{fig:failure}
\end{figure}

\subsection{Evaluation of Text recognition}
In the previous section, two patterns of text recognition were tried: recognition using
CTC and recognition using VNRecognizeText API. Since this report worked on recognition of
handwritten text and made use of the result for auto-complete, following metrics were
designed to measure the goodness of auto-completion.

\begin{itemize}
    \item Omitted Characters Count (OCC) - The gap between how many characters did the writer actually write before
    he found the word he wanted to write within top 10 of the auto-completion candidates
    and the number of characters in the word.
    If it is not found after writing to the end, the score is 0. Higher value of this metric indicates better result.
    \item Cumulative Time for Inference (CTI) - Cumulative time spent on auto-completion until the word that the writer actually
    wants to write is included in the top 10 auto-completion candidates. If it is not found after writing to the end,
    the score cumulative time spent on the recognition at each step. Note that each time writer releases the pen tip from the tablet, recognition
    process runs. Lower value of this metric indicates better result.
\end{itemize}

100 words were selected from the word list in /usr/share/dict/words of Debian GNU/Linux for evaluation.
Table \ref{tab:metrics} shows the performance of both methods on auto-completion. While OCC measure of
VNRecognizeText API show better result than CTC, cumulative time elapsed for inference is almost doubled
compared to that of CTC's.

\begin{table}[htbp]
    \centering
    \begin{tabular}{|l||c|c|} \hline
    method & OCC (mean) & CTI (mean) \\ \hline \hline
    CTC & 1.0102 & 1.1060 \\ \hline
    VNRecognizeText (.accurate) & 3.3405 & 1.9751 \\ \hline
    \end{tabular}
    \caption{Perfromance of CTC and VNRecognizeText API on auto-completion.}
    \label{tab:metrics}
\end{table}

\begin{figure}
    \centering
    \includegraphics[]{images/auto-completion.png}
    \caption{Example of auto-completion showing the top 1 candidate}
    \label{fig:auto-complete}
\end{figure}

Figure \ref{fig:auto-complete} show an example of how auto-completion works.

\subsection{Discussion}

The VNRecognizeText API is superior to CTC in character recognition accuracy,
but in terms of speed it takes about three times longer to infer.
Since auto-completion is an application that requires real-time performance,
a small lag does not provide a good user experience. Therefore,
poor performance in speed can be a problem.

On the other hand, the CTC model shows inference speed that does not the user
feel uncomfortable in actual use, but the inference result is unstable,
and a slight difference in notation greatly affects the inference result.
Figure \ref{fig:unstable} shows an example of unstable inference.

\begin{figure}
    \centering
    \includegraphics{images/table.png}
    \includegraphics{images/able.png}
    \caption{Example of unstable inference}
    \label{fig:unstable}
\end{figure}

This is probably because training of the CTC model seems to be over-fitty
and strongly depends on the dataset used for training.
Therefore, future tasks include reducing the reliance on datasets and
using larger datasets or using data augmentation techniques to improve generalization performance.

\section{Future Work \& Conclusion}
\label{section:conclusion}

The goal of this report is to create an application to recognize sentences
on iOS for documents where handwritten illustrations and characters were mixed.
Although it is imperfect, the heuristic used for region of interest detection successfully
detect a word out of a sentence, and can cut handwritten illustration out from the other
part of document without requreing large amount of computation.
On the other hand, text recognition had a trade-off between speed and accuracy.

Future tasks include improving the accuracy of text recognition without slowing it down.
Another example of future work is to perform more accurate text recognition and ROI detection
using an online method. In auto-completion, prediction using information on surrounding words,
and refinement of the method of presenting complement candidates can also be one of future works.


\bibliographystyle{IEEEtran}
\bibliography{references}

\end{document}
